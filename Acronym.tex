\chapter{Abkürzungsverzeichnis}
% Siehe Paket acronym !!
\begin{acronym}[REKOM]% in [] das längste Acronym (zur optischen ausrichtung)
	
	\acro{CCPS}{cardioscan Checkpoint Software}
	\acro{VT1}{1. Ventilatorische Schwelle}
	\acro{VT2}{2. Ventilatorische Schwelle}
	\acro{RQ}{Respiratorischer Quotient}
	\acro{CO2}[CO\textsubscript{2}]{Kohlenstoffdioxid}
	\acro{O2}[O\textsubscript{2}]{Sauerstoff}
	\acro{VCO2}[\.{V}CO\textsubscript{2}]{Kohlenstoffdioxidabgabe}
	\acro{VO2}[\.{V}O\textsubscript{2}]{Sauerstoffaufnahme}
	\acro{ATP}{Adenosintriphosphat}
	\acro{ADP}{Adenosindiphosphat}
	\acro{Pi}[P\textsubscript{i}]{anorganisches Phosphat}
	\acro{H20}[H\textsubscript{2}O]{Wasser}
	\acro{CrP}{Kreatinphosphat}
	\acro{Cr}{Kreatin}
	\acro{H+}[H\textsuperscript{+}]{Wasserstoff}
	\acro{NADH}{Nicotinamidadenindinukleotid}
	\acro{HLa}{Milchsäure}
	\acro{La-}[La\textsuperscript{-}]{Laktat}
	\acro{LDH}{Laktatdehydrogenase}
	\acro{HCO3-}[HCO\textsubscript{3}\textsuperscript{-}]{Bicarbonat}
	\acro{RCP}{Respiratorischer Kompensationspunkt nach Wasserman (1981)}
	\acro{MLSS}{Maximales Laktat-Steady-State}
	\acro{VE}[\.{V}E]{Ventilation}
	\acro{AMV}{Atemminutenvolumen}
	\acro{FIO2}{Fraktion des inspirierten Sauerstoffs}
	\acro{FEO2}{Fraktion des exspirierten Sauerstoffs}
	\acro{FECO2}{Fraktion des exspirierten Kohlenstoffdioxids}
	\acro{COPD}{chronisch obstruktive Lungenerkrankung}
	\acro{EQCO2}[EQCO\textsubscript{2}]{Kohlenstoffdioxid-Äquivalent}
	\acro{EQO2}[EQO\textsubscript{2}]{Sauerstoff-Äquivalent}
	\acro{W}{Leistung}
	\acro{WL}{"`work load"', Leistung bzw. Belastungsintensität}
	\acro{POW}{Punkt des maximalen Wirkungsgrades nach Hollmann (1959)}
	\acro{EKG}{Elektrokardiogramm}
	\acro{HF}{Herzfrequenz}
	\acro{CSI}{Cardio-Stress-Index}
	\acro{SHIP}{Study of Health in Pomerania}
	\acro{Wmax}[W\textsubscript{max}]{maximale Leistung}
	\acro{Wstart}[W\textsubscript{Start}]{Anfangsbelastung}
	\acro{WHO}{Weltgesundheitsorganisation}
	\acro{BAL}{Bundesausschuss Leistungssport}
	\acro{AF}{Atemfrequenz}
	\acro{CSV}{Comma-separated values}
	\acro{ID}{Identifikator}
	\acro{HFmax}[HF\textsubscript{max}]{maximal erreichte Herzfrequenz}
	\acro{VO2max}[VO\textsubscript{2max}]{absolute maximale Sauerstoffaufnahme}
	\acro{SD}{Standardabweichung}
	\acro{REKOM}{Regeneratives/Kompensatorisches Training}
	\acro{GA1}{Extensives Grundlagentraining}
	\acro{GA2}{Intensives Grundlagentraining}
	\acro{EW}{Entwicklungsbereich}
	
\end{acronym}