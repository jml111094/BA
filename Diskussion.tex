\chapter{Diskussion}

\section{Optimale Methode}

Die VT1-Methoden lieferten im Rahmen der Testmessungen ungenügende Ergebnisse. Die Plots sorgten mit beiden Methoden häufig für Abweichungen zwischen den Ratern und auch die Software definierte oftmals unterschiedliche Werte. Es gestaltet sich bei derartigen Differenzen nicht als sinnvoll, die Algorithmen für VT1 in die \acs{CCPS} zu implementieren. Mit 28 Probanden ist die Anzahl an ausgewerteten Ergebnissen aber sehr gering und belegt keine Validität. Die wenigen großen Differenzen in Abb. \ref{pic:pic23} und Abb. \ref{pic:pic24} haben aufgrund der geringen Anzahl an Personen starken Einfluss auf die Mittelwerte und Standardabweichung, die für die Evaluierung genutzt wurden. Weitere Testprojekte mit größeren Probandengruppen könnten dementsprechend andere Erkenntnisse liefern.\\
Des Weiteren ist die hohe Subjektivität bei der Bestimmung der VT1 durch V-Slope und der VT2 durch \acs{VE}/\acs{VCO2} ein Faktor, der zu einem hohen Maß an Differenzen führt. Eventuell wurde die optische Qualität der 6-Felder-Grafiken durch den zweiten Rater anders wahrgenommen und entspricht nicht der Kategorisierung in Tab. \ref{tab:tabelle7}. Eine größere Menge an Ratern könnte ebenfalls zu gänzlich anderen Mittelwerten und Ergebnissen führen.\\
Die VT2-Methoden sind im direkten Vergleich zusammen mit cardioscans Spiroergometrie-Modul dennoch am genausten. Die algorithmische Bestimmung durch \acs{EQCO2} korreliert stark mit den subjektiven Definitionen der Rater, weshalb angenommen wird, dass die VT2 damit präzise bestimmbar ist. Vorausgesetzt ist hierbei jedoch die korrekte Durchführung der Leistungsdiagnostik unter stetiger Rücksichtnahme auf die aufgeführten Einflussfaktoren. Mit \acs{EQCO2} korrelieren die Ergebnisse außerdem deutlich mehr, als mit RQ = 1.\\
Zwar wurde in der Einleitung darauf hingedeutet, dass VT1 für ein alternatives Trainingszonenmodell im Breitensportbereich sinnvoll wäre. Mit dem Modell von Wilfried Kindermann wurde jedoch eines recherchiert, welches wie der aktuelle Algorithmus der \acs{CCPS} nur von VT2 abhängig ist~\cite{Kindermann.2004}. \acs{REKOM}, \acs{GA1}, \acs{GA2}, \acs{EW} und Leistung sind auch in diesem Modell aufzufinden und ihre Grenzen sind prozentual von der \acs{HF} bei VT2 abhängig. Dieses Modell wurde bereits zum Test in die algorithmischen Plots in Feld 5 implementiert (siehe z.B. Abb. \ref{pic:pic18}). Der alte Algorithmus führte häufig zu Trainingszonen, die sich gegenseitig überlappten, wenn der RQ sehr früh oder gar nicht den Wert eins überschritt und dadurch die VT2 falsch oder gar nicht bestimmt wurde. Beispielsweise ergaben sich somit deutlich zu große Intervalle für den Leistungsbereich von >\SI{40}{\per\minute}, der jedoch per Definition nur im \acs{HF}-Bereich zwischen der VT2 und der \acs{HFmax} liegt. Mithilfe der VT2, die durch das \acs{EQCO2} bestimmt wurde, konnten für alle Personen separierte Trainingszonen definiert werden. Die neue Methode zur Bestimmung der VT2 anhand von \acs{EQCO2} bietet also eine gute Basis zur weiterführenden Forschung und Optimierung des Trainingszonenmodells.

