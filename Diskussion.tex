\chapter{Diskussion}

Die Auswertung des Pilotprojekts wird nun einer methodenkritischen Betrachtung unterzogen und in Bezug auf die in Kapitel 1.3 aufgestellten Forschungsfragen diskutiert. Diese werden anschließend chronologisch beantwortet. Außerdem werden Limitationen und eventuelle Defizite der Methode behandelt und diesbezüglich einige Vorschläge für die Firma cardioscan präsentiert.
\par 
Da mit den vier Methoden für alle Testpersonen Ergebnisse erzielt wurden, kann geschlussfolgert werden, dass der metabolicscan in Verbindung mit der \acs{CCPS} und Fahrradergometern zur Bestimmung der ventilatorischen Schwellen genutzt werden kann. Während der Messungen wurden keine Störungen oder Fehler der verbauten Sensoren festgestellt. Die maximal gemessene Atemfrequenz bei den Testmessungen betrug \SI{49,55}{\per\minute} und lag damit innerhalb der Grenzen des \acs{O2}-Sensors. Mit \SI{136,27}{\litre\per\minute} lag auch die maximal gemessene \acs{VE} unterhalb des Maximums des Flowsensors. Der metabolicscan wurde vor dem Projekt mithilfe eines Lungensimulators bei Atemfrequenzen zwischen \SIlist{6;50}{\per\minute} und unterschiedlichen Gaskonzentrationen kalibriert, weswegen ausgeschlossen werden kann, dass die Sensoren fehlerhaft waren. Allerdings konnten Anfälligkeiten für probanden-, -anwender- sowie umweltbedingte Fehler festgestellt werden.

\section{Potentielle Fehlerquellen}

\subsection{Probandenbedingte Faktoren}

Bei der Spiroergometrie können Probanden die Ergebnisse negativ beeinflussen, indem sie ihre Atmung durch die ungewohnten Bedingungen stark verändern, sodass diese unphysiologisch wird. Bei wenigen Personen konnte beobachtet werden, dass sie zu Beginn der Leistungsdiagnostik eine recht hohe \acs{VE} besaßen und die \acs{AF} schnell zunahm. Dies lag vor allem am Mundstück und dem Bakterienfilter. Durch diesen erhöht sich der Atemwiderstand, dessen Intensität von jeder Person subjektiv anders wahrgenommen wird. Durch das Empfinden eines erhöhten Atemwiderstandes neigten die Probanden dazu, tiefer und gleichzeitig schneller zu atmen. Dadurch stieg die \acs{VE} relativ zur \acs{VO2} höher, da der Körper zu diesem Zeitpunkt noch nicht die große Menge an aufgenommenem \acs{O2} verwerten konnte. Des Weiteren können Fehler entstehen, wenn Personen das Mundstück nicht rechtzeitig zu Munde führen. Die Dauer der Messphase beträgt stets \SI{30}{\second}. Zu Beginn einer Leistungsdiagnostik bei geringer Leistung besitzen die meisten Menschen eine niedrige \acs{AF} von beispielsweise \SI{12}{\per\minute}, also $6$ Atemzügen in \SI{30}{\second}, oder weniger. Kommt es zu Problemen mit dem Mundstück, wie bei Probandin 19w, werden während einer Messung durch die algorithmischen Filter weniger Atemzüge erfasst, sodass die gemittelte Atemfrequenz und dadurch auch die \acs{VE} kleiner wird. Das wiederum kann zu veränderten Plots führen, wenn auf einen verfälscht kleineren in der nächsten Stufe ein normaler Mittelwert folgt, wodurch die Steigung zwischen diesen zwei Punkten verfälscht wird und nicht mehr den ventilatorischen Reaktionen des Körpers entspricht. Zuletzt muss auf die Ernährung unmittelbar vor einer Spiroergometrie geachtet werden (siehe Kapitel 2.1.2). Zwar wird zukünftig der RQ nicht mehr für die Schwellenbestimmung verwendet, jedoch sollte der Grund-Laktat-Gehalt nicht durch erhöhte Zuckerzufuhr erhöht werden, um die Ausgangsbedingungen zu optimieren.

\subsection{Anwenderbedingte Faktoren}

Durch den Anwender der Spiroergometrie können ebenfalls Fehler verursacht werden. Beispielsweise kann oben genannter zeitlicher Verzug auch entstehen, wenn der Anwender dem Probanden das Mundstück zu spät reicht, sodass die Atemzugerfassung ebenfalls verfälscht wird. Schon eine falsche Vorbereitung der Belastungsphase kann zu Fehlern führen, wenn z.B. die Sattelhöhe nicht korrekt justiert wurde oder ein unpassendes Belastungsprotokoll bestimmt wurde. Es ist wichtig, dass der Anwender durch das Interview den Trainingszustand einer Person korrekt einschätzen zu können und entsprechende Anpassungen am Protokoll vorzunehmen. Allerdings ist hier auch die deutliche Kommunikation mit dem Probanden zu berücksichtigen.

\subsection{Umweltbedingte Faktoren}

Wie in Kapitel 2.1.2 erwähnt, sind die Messbedingungen für eine Respirationsanalyse einzuhalten. Raumtemperatur und \acs{CO2}-Gehalt in der Luft wirken sich auf \acs{HF} und \acs{La-}-Kinetik aus, was ebenfalls im Falle einer Nichtbeachtung zu Fehlern führt. Jedoch wurde der Raum vor jedem Test belüftet und die Temperatur im Toleranzbereich zwischen \SIrange{18}{22}{\degreeCelsius} gehalten. Die \acs{CO2}-Belastung der Atemluft in $10^{-6}$ (parts per million) wurde durch ein Messgerät sehr genau überprüft, um sicherzustellen, dass die Kalibrierung des metabolicscan vor jedem Test auf denselben Grundbedingungen basierte.

\section{Problematik von RQ = 1}

Tab. \ref{tab:tabelle6} beweist sehr deutlich, dass die Methode RQ = 1 zur alleinigen Bestimmung der VT2 nicht valide ist. Bei 9 von 28 Personen stieg der Wert nicht über eins, sodass VT2 gar nicht bestimmt werden konnte. Hier ließe sich diskutieren, ob diese Personen tatsächlich kardiorespiratorisch ausbelastet waren, oder den Test wegen muskulärer Erschöpfung beendet haben. Dennoch konnte bei diesen neun Probanden mit \acs{EQCO2} sowie \acs{VE}/\acs{VCO2} VT2 bestimmt werden. Bei zehn Personen, bei denen mit der Referenzmethode eine VT2 definiert werden konnte, weichen die Werte jedoch von den übrigen Methoden mit \SI{10}{\per\minute} oder mehr ab. Darüber hinaus wurden für den MATLAB-Algorithmus bereits Ausgleichskurven für den RQ implementiert. Diese sorgen jedoch nicht dafür, dass die Mehrheit der Messergebnisse mit den übrigen Methoden vergleichbar wird. Der Algorithmus wird künftig aus der \acs{CCPS} entfernt werden.

\section{Optimale Methode}

Die VT1-Methoden lieferten im Rahmen der Testmessungen ungenügende Ergebnisse. Die Plots sorgten mit beiden Methoden häufig für Abweichungen zwischen den Ratern und auch die Software definierte oftmals unterschiedliche Werte. Es gestaltet sich bei derartigen Differenzen nicht als sinnvoll, die Algorithmen für VT1 in die \acs{CCPS} zu implementieren. Die VT2-Methoden sind im direkten Vergleich zusammen mit cardioscans Spiroergometrie-Modul am genausten. Die algorithmische Bestimmung durch \acs{EQCO2} korreliert stark mit den subjektiven Definitionen der Rater, weshalb angenommen werden kann, dass die VT2 damit präzise bestimmbar ist. Vorausgesetzt ist hierbei jedoch die korrekte Durchführung der Leistungsdiagnostik unter stetiger Rücksichtnahme auf die aufgeführten Einflussfaktoren. Mit \acs{EQCO2} korrelieren die Ergebnisse deutlich mehr, als mit RQ = 1.\\
Zwar wurde in der Einleitung darauf hingedeutet, dass VT1 für ein alternatives Trainingszonenmodell im Breitensportbereich sinnvoll wäre. Mit dem Modell von Wilfried Kindermann wurde jedoch eines recherchiert, welches wie der aktuelle Algorithmus der \acs{CCPS} nur von VT2 abhängig ist~\cite{Kindermann.2004}. \acs{REKOM}, \acs{GA1}, \acs{GA2}, \acs{EW} und Leistung sind auch in diesem Modell aufzufinden und ihre Grenzen sind prozentual von der \acs{HF} bei VT2 abhängig. Dieses Modell wurde bereits zum Test in die algorithmischen Plots in Feld 5 implementiert (siehe z.B. Abb. \ref{pic:pic18}). Die neue Methode zur Bestimmung der VT2 anhand von \acs{EQCO2} bietet also eine gute Basis für die Trainingszonendefinition.

\section{Limitation}

Das Pilotprojekt war limitiert durch eine relativ kleine Teilnehmerzahl und dadurch sehr ungleiche Verteilung der unterschiedlichen Probandengruppen. Um die Aussagekraft der Ergebnisse zu erhöhen und diese beispielsweise besser mit der HUNT 3 Studie vergleiche zu können, wäre eine deutlich größere Probandengruppe notwendig gewesen. Auch die Quantität an validen Referenzdaten ist nicht sonderlich ausgeprägt. Solche wären für einen Vergleich der Ergebnisse nützlich gewesen. 

\section{Fazit und Handlungsempfehlung}

Ausgehend von den Ergebnissen dieser Arbeit wurde mit \acs{EQCO2} eine neue Methode für die VT2-Bestimmung erarbeitet, welche zukünftig als Algorithmus die Basis für die Auswertung durch die \acs{CCPS} darbieten wird. In Verbindung mit dem Trainingszonenmodell nach Kindermann können mit dieser Methode die Ziele der Firma cardioscan umgesetzt werden. Das Leistungsdiagnostik-Setup bietet jedoch Optimierungsmöglichkeiten. Zum ersten wären eventuell Alternativen zu dem Mundstück, z.B. in Form einer Maske sinnvoll, da die Ergonomie des Mundstücks eine gewisse Problematik darstellt. Hier könnten weitere künftige Testmessungen oder Studien mit alternativen Komponenten durchgeführt werden. Zum Zweiten gestaltet sich die Mittelung der Messwerte über die Gesamtanzahl an Atemzügen pro Stufe als nicht optimal, da dadurch die vorangegangen beschriebenen Probleme auftreten können. Eine gänzliche "`breath-by-breath"'-Auswertung jedes Atemzugs sollte wegen der zu großen Datenmenge zwar auch nicht in Betracht bezogen werden, jedoch könnten alternative Mittelungsverfahren (z.B. die "`gleitende"' Mittelung~\cite{Kroidl.2015}) getestet werden.\\
Die algorithmische Auswertung könnte für die anschließende Interpretation optimiert werden, indem z.B. ein Algorithmus programmiert wird, der ein Abflachen der \acs{HF}-Kurve zum Ende einer Leistungsdiagnostik überprüft. Dies könnte den Anwender dabei unterstützen, zu bewerten, ob eine Person tatsächlich kardiorespiratorisch ausbelastet war (siehe Kap. 2.2.1).\\
Zum Abschluss bleibt zu erwähnen, dass, trotz algorithmischer und software-gestützter Auswertung einer Spiroergometrie, eine Interpretation der Ergebnisse seitens einer Person mit Fachwissen unabdingbar ist. Die gezeigten Übersichtsplots, anhand derer die Schwellen bestimmt wurden, bieten lediglich die Grundlage und müssen in ein optimiertes Interface der \acs{CCPS} implementiert werden. Die komplette Thematik der Atemgasanalyse ist im Allgemeinen sehr komplex und sollte daher nicht von ungeschultem Personal behandelt werden. cardioscan betreibt für alle diagnostischen Anwendungen eine Academy, welche in regelmäßigen Intervallen Seminare und Workshops durch Schulungsreferenten und -referentinnen beim Kunden durchführen lässt. Bei diesen Seminaren sollten auch die Feinheiten der Vorbereitung behandelt werden. Die Software könnte die Anwender auch auf diesem Wege unterstützen, indem z.B. der Timer im Interface durch einen Hinweis ergänzt wird, der mit einer gewissen Vorlaufzeit daran erinnert, das Mundstück zur Messung vorzubereiten. Damit könnten Messfehler reduziert werden.


\cleardoublepage
\nocite{*}

%\printnomenclature

%\bibliography{Referenzen}
\printbibliography