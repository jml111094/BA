\chapter{Diskussion}

Die Auswertung des Pilotprojekts wird nun einer methodenkritischen Betrachtung unterzogen und in Bezug zu den aufgestellten Forschungsfragen diskutiert. 
%
\begin{tabbing}
	Ist die Bestimmung der Schwellen mit dem metabolicscan realisierbar?\\
	Welche Methode ist optimal?\\
	Kann die VT2 mit den neuen Methoden genauer bestimmt werden, als mit RQ~=~1?
\end{tabbing}
%
Außerdem werden Limitationen und eventuelle Defizite der Durchführung behandelt und diesbezüglich einige Vorschläge für die Firma cardioscan präsentiert.\\[1em]
Da mit den vier Methoden für alle Testpersonen Ergebnisse erzielt wurden, kann geschlussfolgert werden, dass der metabolicscan in Verbindung mit der \acs{CCPS} und Fahrradergometern zur Bestimmung der ventilatorischen Schwellen genutzt werden kann. Während der Messungen wurden keine Störungen oder Fehler der verbauten Sensoren festgestellt. Die maximal gemessene Atemfrequenz bei den Testmessungen betrug \SI{52,5}{\per\minute} und liegt damit innerhalb der Grenzen des \acs{O2}-Sensors. Mit \SI{136,27}{\litre\per\minute} befindet sich auch die maximal gemessene \acs{VE} unterhalb des Maximums des Flowsensors. Der metabolicscan wurde vor dem Projekt mithilfe eines Lungensimulators bei Atemfrequenzen zwischen \SIlist{6;50}{\per\minute} und unterschiedlichen Gaskonzentrationen kalibriert, weswegen ausgeschlossen werden kann, dass die Sensoren fehlerhaft waren. Zudem wurde jeder bisher produzierte metabolicscan geprüft und mithilfe der Ergebnisse eine Ausgleichskurve für die Flowmessung entwickelt, wie in die Gerätekalibrierung zu Beginn einer Messung implementiert ist. Dennoch können Abweichungen in der Messtechnik nicht vollkommen ausgeschlossen werden. Zusätzlich konnten Anfälligkeiten für probanden-, -anwender- sowie umweltbedingte Fehler festgestellt werden.

\section{Potentielle Fehlerquellen}

\subsection{Probandenbedingte Faktoren}

Bei der Spiroergometrie können Probanden die Ergebnisse negativ beeinflussen, indem sie ihre Atmung durch die ungewohnten Bedingungen stark verändern, sodass diese unphysiologisch wird. Bei wenigen Personen konnte beobachtet werden, dass sie zu Beginn der Leistungsdiagnostik eine recht hohe \acs{VE} besaßen und die \acs{AF} schnell zunahm. Dies lag vor allem am Mundstück und dem Bakterienfilter. Durch diesen erhöht sich der Atemwiderstand, dessen Höhe von jeder Person subjektiv anders wahrgenommen wird. Dadurch neigten die Probanden dazu, tiefer und gleichzeitig schneller zu atmen. Infolgedessen stieg die \acs{VE} relativ zur \acs{VO2} höher, da der Körper zu diesem Zeitpunkt noch nicht die große Menge an aufgenommenem \acs{O2} verwerten konnte. Des Weiteren können Fehler entstehen, wenn Personen das Mundstück nicht rechtzeitig zu Munde führen. Die Dauer der Messphase beträgt stets \SI{30}{\second}. Zu Beginn einer Leistungsdiagnostik bei geringer Leistung ist die \acs{AF} bei den meisten gesunden Menschen noch recht niedrig. Die durchschnittliche \acs{AF} eines Erwachsenen in Ruhe beträgt ca. \SIrange{7}{20}{\per\minute}~\cite{Larsen.2017}. Kommt es zu Problemen mit dem Mundstück, sodass gewisse Schwellwerte bei der Atemzugerkennung der Software nicht überschritten und weniger Atemzüge erfasst werden, wird die gemittelte \acs{AF} und dadurch auch die \acs{VE} kleiner. Das wiederum kann zu veränderten Plots führen, wenn auf einen kleineren Mittelwert in der nächsten Stufe ein normaler folgt, wodurch die Steigung zwischen diesen zwei Punkten verfälscht wird und nicht mehr den ventilatorischen Reaktionen des Körpers entspricht. Zuletzt muss die Ernährung unmittelbar vor einer Spiroergometrie berücksichtigt werden (siehe Kapitel 2.1.2). Zwar wird zukünftig der RQ nicht mehr für die Schwellenbestimmung verwendet, jedoch sollte der Grund-Laktat-Gehalt nicht durch erhöhte Zuckerzufuhr erhöht werden, um die Ausgangsbedingungen in Bezug auf die \acs{VCO2} zu normalisieren.

\subsection{Anwenderbedingte Faktoren}

Durch den Anwender der Spiroergometrie können ebenfalls Fehler verursacht werden. Beispielsweise kann oben genannter zeitlicher Verzug auch entstehen, wenn der Anwender dem Probanden das Mundstück zu spät reicht, sodass die Atemzugerfassung ebenfalls verfälscht wird. Schon eine falsche Vorbereitung der Belastungsphase kann zu Fehlern führen, wenn z.B. die Sattelhöhe nicht korrekt justiert oder ein unpassendes Belastungsprotokoll bestimmt wurde. Es ist wichtig, dass der Anwender im Vorwege den Trainingszustand einer Person korrekt einschätzen kann, um entsprechende Anpassungen am Protokoll vorzunehmen. Allerdings ist hier auch die deutliche Kommunikation mit dem Probanden zwingend erforderlich.

\subsection{Umweltbedingte Faktoren}

Wie in Kapitel 2.1.2 erwähnt, sind die Messbedingungen für eine Respirationsanalyse einzuhalten. Raumtemperatur und \acs{CO2}-Gehalt in der Luft wirken sich auf die \acs{HF} und \acs{La-}-Kinetik aus, was ebenfalls im Falle einer Nichtbeachtung zu Fehlern führt. Jedoch wurde der Raum vor jedem Test belüftet und die Temperatur im Toleranzbereich zwischen \SIrange{18}{22}{\degreeCelsius} gehalten. Die \acs{CO2}-Belastung der Atemluft in $10^{-6}$ (parts per million) wurde durch ein Messgerät sehr genau überprüft, um sicherzustellen, dass die Kalibrierung des metabolicscan vor jedem Test auf denselben Grundbedingungen basierte.

\section{Optimale Methode}

Die VT1-Methoden lieferten im Rahmen der Testmessungen ungenügende Ergebnisse. Die Plots sorgten mit beiden Methoden häufig für Abweichungen zwischen den Ratern und auch die Software definierte oftmals unterschiedliche Werte. Es gestaltet sich bei derartigen Differenzen nicht als sinnvoll, die Algorithmen für VT1 in die \acs{CCPS} zu implementieren. Mit 28 Probanden ist die Anzahl an ausgewerteten Ergebnissen aber sehr gering und belegt keine Validität. Die wenigen großen Differenzen in Abb. \ref{pic:pic23} und Abb. \ref{pic:pic24} haben aufgrund der geringen Anzahl an Personen starken Einfluss auf die Mittelwerte und Standardabweichung, die für die Evaluierung genutzt wurden. Weitere Testprojekte mit größeren Probandengruppen könnten dementsprechend andere Erkenntnisse liefern.\\
Des Weiteren ist die hohe Subjektivität bei der Bestimmung der VT1 durch V-Slope und der VT2 durch \acs{VE}/\acs{VCO2} ein Faktor, der zu einem hohen Maß an Differenzen führt. Eventuell wurde die optische Qualität der 6-Felder-Grafiken durch den zweiten Rater anders wahrgenommen und entspricht nicht der Kategorisierung in Tab. \ref{tab:tabelle7}. Eine größere Menge an Ratern könnte ebenfalls zu gänzlich anderen Mittelwerten und Ergebnissen führen.\\
Die VT2-Methoden sind im direkten Vergleich zusammen mit cardioscans Spiroergometrie-Modul dennoch am genausten. Die algorithmische Bestimmung durch \acs{EQCO2} korreliert stark mit den subjektiven Definitionen der Rater, weshalb angenommen wird, dass die VT2 damit präzise bestimmbar ist. Vorausgesetzt ist hierbei jedoch die korrekte Durchführung der Leistungsdiagnostik unter stetiger Rücksichtnahme auf die aufgeführten Einflussfaktoren. Mit \acs{EQCO2} korrelieren die Ergebnisse außerdem deutlich mehr, als mit RQ = 1.\\
Zwar wurde in der Einleitung darauf hingedeutet, dass VT1 für ein alternatives Trainingszonenmodell im Breitensportbereich sinnvoll wäre. Mit dem Modell von Wilfried Kindermann wurde jedoch eines recherchiert, welches wie der aktuelle Algorithmus der \acs{CCPS} nur von VT2 abhängig ist~\cite{Kindermann.2004}. \acs{REKOM}, \acs{GA1}, \acs{GA2}, \acs{EW} und Leistung sind auch in diesem Modell aufzufinden und ihre Grenzen sind prozentual von der \acs{HF} bei VT2 abhängig. Dieses Modell wurde bereits zum Test in die algorithmischen Plots in Feld 5 implementiert (siehe z.B. Abb. \ref{pic:pic18}). Der alte Algorithmus führte häufig zu Trainingszonen, die sich gegenseitig überlappten, wenn der RQ sehr früh oder gar nicht den Wert eins überschritt und dadurch die VT2 falsch oder gar nicht bestimmt wurde. Beispielsweise ergaben sich somit deutlich zu große Intervalle für den Leistungsbereich von >\SI{40}{\per\minute}, der jedoch per Definition nur im \acs{HF}-Bereich zwischen der VT2 und der \acs{HFmax} liegt. Mithilfe der VT2, die durch das \acs{EQCO2} bestimmt wurde, konnten für alle Personen separierte Trainingszonen definiert werden. Die neue Methode zur Bestimmung der VT2 anhand von \acs{EQCO2} bietet also eine gute Basis zur weiterführenden Forschung und Optimierung des Trainingszonenmodells.

\section{Problematik von RQ = 1}

Tab. \ref{tab:tabelle6} beweist sehr deutlich, dass die Methode RQ = 1 zur alleinigen Bestimmung der VT2 nicht valide ist. In Kapitel 1.3 wurde erwähnt, dass der RQ einst häufig zu früh den Wert eins überschritt und die Schwellenbestimmung dadurch fehlerhaft wurde. Dieser Fall trat bei den Ergebnissen der 28 Tests nicht auf. Dies ist jedoch dadurch begründet, dass durch cardioscans Entwicklung Ausgleichsgeraden in den Algorithmus implementiert wurden. Darüber hinaus wurde vor Start des Projektes ein Befehl aus der Ansteuerung der metabolicscan entfernt, der bewirkt hatte, dass dieser sich zwischen den einzelnen Belastungsstufen rekalibriert. Da jedoch bei einer Spiroergometrie durch den Probanden mehr \acs{CO2} exspiriert wird und die Räumlichkeiten währenddessen nicht belüftet werden, um Störungen des Flowsensors zu vermeiden, steigt entsprechend die \acs{CO2}-Konzentration der Luft. Dies konnte gemessen werden und wirkte sich somit auf die Kalibrierung des Geräts aus, sodass ungewollte Drifts in der RQ-Kurve entstanden.\\
Bei 9 von 28 Personen stieg der Wert stattdessen gar nicht über eins, sodass die VT2 gar nicht bestimmt werden konnte. Hier ließe sich diskutieren, ob diese Personen tatsächlich kardiorespiratorisch ausbelastet waren, oder den Test wegen muskulärer Erschöpfung beendet haben. Dennoch konnte bei diesen neun Probanden mit \acs{EQCO2} sowie \acs{VE}/\acs{VCO2} die VT2 bestimmt werden. Bei zehn Personen, bei denen mit der Referenzmethode eine VT2 definiert werden konnte, weichen die Werte jedoch von den übrigen Methoden mit \SI{10}{\per\minute} oder mehr ab. Darüber hinaus wurden für den MATLAB-Algorithmus bereits Ausgleichskurven für den RQ implementiert. Diese sorgen jedoch nicht dafür, dass die Mehrheit der Messergebnisse mit den übrigen Methoden vergleichbar wird. Der Algorithmus wird künftig aus der \acs{CCPS} entfernt und durch die genauere \acs{EQCO2}-Methode ersetzt.

\section{Limitation}

Das Pilotprojekt war limitiert durch eine relativ kleine Teilnehmerzahl und dadurch sehr ungleiche Verteilung der unterschiedlichen Probandengruppen. Um die Aussagekraft der Ergebnisse zu erhöhen und diese beispielsweise besser mit der HUNT 3 Studie vergleichen zu können, wäre eine deutlich größere Probandengruppe notwendig gewesen. Die Evaluierung war auch durch eine kleine Menge an wertenden Personen und einen gleichzeitigen Mangel an validen Referenzwerten eingeschränkt. Generell sind Normwerte im Bereich der Sportwissenschaft schwer zu definieren, da die Individualität an Zuständen zu groß ist. Dies bedeutet, dass bei der Analyse der individuellen Leistungsfähigkeit noch sensibler auf eventuelle Einflussfaktoren eingegangen werden muss, als beispielsweise bei einer rein medizinischen Lungenfunktionsmessung, der einige Normbereiche für z.B. das \ac{FEV1} oder \ac{FVC} zu Grunde liegen, welche anatomisch abgegrenzt sind.

\section{Fazit und Handlungsempfehlung}

Ausgehend von den Ergebnissen dieser Arbeit wurde mit dem \acs{EQCO2} eine neue Methode für die VT2-Bestimmung erarbeitet, welche zukünftig als Algorithmus die Basis für die Auswertung durch die \acs{CCPS} darbieten kann. In Verbindung mit dem Trainingszonenmodell nach Wilfried Kindermann können mit dieser Methode die Ziele der Firma cardioscan umgesetzt werden. Das Leistungsdiagnostik-Setup bietet jedoch Optimierungsmöglichkeiten. Zum Ersten wären eventuell Alternativen zu dem Mundstück, z.B. in Form einer Maske sinnvoll, da die Ergonomie des Mundstücks moniert wurde. Hier könnten weitere Tests mit alternativen Komponenten durchgeführt werden. Zum Zweiten gestaltet sich die Mittelung der Messwerte über die Gesamtanzahl an Atemzügen pro Stufe als nicht optimal, da dadurch die vorangegangen beschriebenen Verfälschung der Plots auftreten können. Eine gänzliche "`breath-by-breath"'-Auswertung jedes Atemzugs sollte wegen der zu großen Datenmenge zwar auch nicht in Betracht bezogen werden, jedoch könnten alternative Mittelungsverfahren (z.B. die "`gleitende"' Mittelung~\cite{Kroidl.2015}) getestet werden.\\
Die algorithmische Auswertung könnte für die anschließende Interpretation optimiert werden, indem z.B. ein Algorithmus programmiert wird, der ein Abflachen der \acs{HF}-Kurve zum Ende einer Leistungsdiagnostik überprüft. Dies könnte den Anwender dabei unterstützen, zu bewerten, ob eine Person tatsächlich ausbelastet war. In der HUNT 3 Studie wurde ermittelt, bei wie viel Prozent der \acs{HFmax} die VT2 einer Person erreicht worden war. Eine abflachende \acs{HF}-Kurve wäre ein Indikator für das Erreichen der \acs{HFmax}. Durch einen Vergleich der \acs{HF} bei der VT2 und der \acs{HFmax} könnten Ergebnisse zusätzlich evaluiert werden.\\
Zum Abschluss bleibt zu erwähnen, dass, trotz algorithmischer Auswertung einer Spiroergometrie, eine Interpretation der Ergebnisse seitens einer Person mit Fachwissen unabdingbar ist. Die gezeigten Grafiken, anhand derer die Schwellen bestimmt wurden, bieten lediglich die Grundlage und müssen in ein optimiertes Interface der \acs{CCPS} implementiert werden. Die komplette Thematik der Atemgasanalyse ist im Allgemeinen sehr komplex und sollte daher nicht von ungeschultem Personal behandelt werden. cardioscan betreibt für alle diagnostischen Anwendungen eine Academy, welche in regelmäßigen Intervallen Seminare und Workshops durch Schulungsreferenten und -referentinnen beim Kunden durchführen lässt. Bei diesen Seminaren sollten auch die Feinheiten der Vorbereitung behandelt werden. Die Software könnte die Anwender auch auf diesem Wege unterstützen, indem z.B. der Timer im Interface durch einen Hinweis ergänzt wird, der daran erinnert, das Mundstück zur Messung vorzubereiten, um Messfehler zu reduzieren.


\nocite{*}

%\printnomenclature

%\bibliography{Referenzen}
\printbibliography