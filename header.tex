\usepackage{hyphsubst}
\HyphSubstIfExists{ngerman-x-latest}{%
	\HyphSubstLet{ngerman}{ngerman-x-latest}}{}
\usepackage[ngerman]{babel}
\usepackage[T1]{fontenc}
\usepackage[utf8]{inputenc}
\usepackage{textcomp}
\usepackage{lmodern}
\usepackage{times}
\fontfamily{ptm}\selectfont
\KOMAoptions{DIV=last}
\usepackage[final,babel]{microtype}

\setkomafont{disposition}{\normalcolor\bfseries}

\usepackage{chngcntr}
\counterwithout{figure}{chapter}
\counterwithout{table}{chapter}

\usepackage{etoolbox}
\usepackage{tabularx}
\usepackage{tabulary}
\usepackage{colortbl}
\usepackage{subcaption}
\captionsetup[subfigure]{list=true, font=small, labelfont=bf, 
	labelformat=brace, position=top}
\usepackage{multirow}
\usepackage{enumitem}
\newlist{titemize}{itemize}{1}																% neue Listenumgebung für Tabellen 
\setlist[titemize]{nosep, label=\textbullet, after=\strut, align=left, leftmargin=*}

\DeclareMicrotypeSet*[tracking]{my}%
  { font = */*/*/sc/* }%
\SetTracking{ encoding = *, shape = sc }{ 45 }


\usepackage[inner=34mm, outer=30mm]{geometry}
\usepackage[automark]{scrlayer-scrpage}
\pagestyle{scrheadings}																		% lebendiger Seitentitel

\raggedbottom
\setlength{\parindent}{0em}
\setlength{\parskip}{1ex}

\newcommand{\mydate}{\today}

\usepackage{amsmath}																		% erweiterter Mathematiksatz
\usepackage{amssymb}																		% weitere Mathematiksymbole
\renewcommand\theequation{\thechapter-\arabic{equation}}

\usepackage{pdflscape} 
\usepackage[tight,nice]{units}																% ermöglicht Angabe physikal. Größen
\usepackage[version-1-compatibility]{siunitx}
\sisetup{detect-family, locale=DE}
\usepackage{booktabs}

\usepackage{dcolumn}																		% für gezielte Ausrichtung von Zahlen in Tabellen

\usepackage{url}																			% für url-Angaben

\usepackage{float}
\usepackage{wasysym}

\usepackage{relsize}																		% Schriftgrüßen nicht absolut, sondern relativ angeben
\usepackage{setspace}																		% Zeilenabstand

\usepackage[margin=10pt,font=small,labelfont=bf,labelsep=endash,format=hang,figurename={Abb.}, tablename={Tab.}
]{caption}																					% für schönere Unterschriften bei Abbildungen und Tabellen

\usepackage{calc}
%\usepackage[printonlyused]{acronym}

\usepackage[section]{placeins}																% Einfügen einer "FloatBarrier"

\usepackage[final]{graphicx}																% Paket zum Einbinden von Bildern, Option "final" hebt die globale "draft" bei Abbildungen auf

\usepackage{textpos}																		% für millimetergenaue Paltzierung von Text(Grafik-)blöcken
\usepackage[update]{epstopdf}

\usepackage{lipsum}																			% Paket für "Lorem Ipsum" Paragraphen
\usepackage{pdfpages}
\usepackage[]{todonotes}
%\usepackage{pdfcomment}

\usepackage{scrhack}																		% Damit keine Warnung \float@addtolists auftritt (bei verwendung von listings)
\usepackage{listings}
\lstset{ %
language=Matlab,
frame=single,
tabsize=4,
breaklines=true,
breakatwhitespace=true,
basicstyle=\footnotesize,
}
\renewcommand{\lstlistingname}{Quellcode} 													% Umbennen 'Listing' nach 'Quellcode'
\usepackage{rotating}

\usepackage[german,intoc,refpage]{nomencl}
\let\abbr\nomenclature																		% Befehl nomenclature umbenennen in abbr
\renewcommand{\nomname}{Symbolverzeichnis}													% Abkürzungsverzeichnis soll Symbolverzeichnis heißen
\setlength{\nomlabelwidth}{.20\hsize}
\renewcommand{\nomlabel}[1]{#1 \dotfill}													% Punkte zw. Abkürzung und Erklärung
\setlength{\nomitemsep}{-\parsep}															% Zeilenabstände verkleinern
\makenomenclature																			% Verzeichnis erstellen



\author{Julian-Marvin}
\title{Evaluierung von Methoden zur ventilatorischen Schwellenbestimmung in der Spiroergometrie}
\subject{Bachelorarbeit}
\date{\mydate}
\publishers{Fachhochschule Lübeck\\ --- Fachbereich Angewandte Naturwissenschaften}

\usepackage[nomain,nonumberlist,nopostdot,nogroupskip,style=super,acronym,xindy,toc]{glossaries}
\newcommand{\myglsgen}[1]{%
	\glsdoifexists{#1}%
	{%
		\ifglsused{#1}{%
			\acrshort{#1}%
		}% 
		{%
			\glsuseri{#1} (\acrshort{#1})%
			\glsunset{#1}%
		}%
	}%	
}%
\setacronymstyle{long-short}
\glsaddall
\makeglossaries
\usepackage[xindy]{imakeidx}
\makeindex

\usepackage[autostyle=true,german=quotes]{csquotes}
\usepackage[
backend=biber,
style=iso-numeric,
autolang=other,
bibencoding=UTF8,
maxnames=3,
hyperref
]{biblatex}

\DefineBibliographyStrings{ngerman}{andothers={et\ al\adddot}}
\addbibresource{Referenzen.bib}

\apptocmd{\UrlBreaks}{\do\f\do\m}{}{}
\setcounter{biburllcpenalty}{9000}% Kleinbuchstaben
\setcounter{biburlucpenalty}{9000}% Großbuchstaben

\usepackage{hyperref}
\hypersetup{
	unicode=false,
	pdftoolbar=true,
	pdfmenubar=true,
	pdffitwindow=false,
	pdfstartview={FitH},
	pdfborder={Thema der Arbeit},
	pdftitle={Thema der Arbeit},
	pdfauthor={Julian-Marvin Lütten},
	pdfsubject={Bachelorthesis},
	pdfnewwindow=true,
	colorlinks=true,
	linkcolor=black,
	citecolor=black,
	filecolor=black,
	urlcolor=black,
	breaklinks=true
}

\endinput

