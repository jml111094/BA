\chapter{Hauptkapitel}


\section{Erster Abschnitt}


%\lipsum[1-2]

Im \ac{RD}\todo{Erklären, was \ac{RD} ist} und auf \ac{IS} sowie in der Anästhesie  ist die Pulsoxymetrie Teil des Standardmonitorings des Patienten. Bei Frühgeburten wird zur weiteren häuslichen Überwachung oft ein Überwachungsmonitor eingesetzt, der die Atemfrequenz, die Sauerstoffsättigung und den Puls misst.

Bei der Überwachung von Frühgeborenen in der neonatalen Intensivmedizin kommt häufig auch die duale Pulsoxymetrie (rechts - links) zum Einsatz, um bei diagnostiziertem persistierenden Ductus arteriosus den Unterschied zwischen präduktaler und postduktaler Sauerstoffsättigung im zeitlichen Verlauf zu erfassen.

Die Pulsoxymetrie%\pdfcomment{Schreib mal was rein!}
 wird im \ac{RD} bei Flügen in große Höhen eingesetzt, um so durch Selbstkontrolle einer Hypoxie vorbeugen zu können.\todo[inline]{Noch genauer auf \ac{RD} eingehen!}

\begin{figure}
	\centering
		\missingfigure{Hier fehlt die Beispielabbildung!}
	\caption{Dies ist eine wichtige Beispielabbildung.}
	\label{fig:bsp}
\end{figure}


Auch im Bereich des Höhenbergsteigens werden immer öfter Pulsoxymeter verwendet um eine drohende Höhenkrankheit frühzeitig zu erkennen.


\section{Zweiter Abschnitt}


\lipsum[3-4]


\chapter{Bemerkungen}


\lipsum[5]


